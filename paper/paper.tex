%\documentclass{recpad2k}
\documentclass[extendedabs]{recpad2k}

%% Enter your paper number here for the review copy
\recpadreviewcopy{??}

\title{Clustering DNA sequences by relative compression}

% Enter the paper's authors in order
\addauthor{Morteza Hosseini}{seyedmorteza@ua.pt}{1}
\addauthor{Diogo Pratas}{pratas@ua.pt}{1}
\addauthor{Armando J. Pinho}{ap@ua.pt}{1}

% Enter the institutions
\addinstitution{
 IEETA/DETI,\\
 University of Aveiro
}

\runninghead{Student, Prof, Collaborator}{RECPAD Author Guidelines}

% Any macro definitions you would like to include
% These are not defined in the style file, because they don't begin
% with \bmva, so they might conflict with the user's own macros.
% The \bmvaOneDot macro adds a full stop unless there is one in the
% text already.
\def\eg{\emph{e.g}\bmvaOneDot}
\def\Eg{\emph{E.g}\bmvaOneDot}
\def\etal{\emph{et al}\bmvaOneDot}

%------------------------------------------------------------------------- 
% Document starts here
\begin{document}

\maketitle

\begin{abstract}
   In this paper, we present a method for clustering DNA sequences, using relative compression. Tested on 30 different sequences, we could successfully classify them into three groups. The results show that two groups of Actinopterygii and Chondrichthyes, that are both fishes, are more similar to each other, compared to Mammalia group.
\end{abstract}

%------------------------------------------------------------------------- 
\section{Introduction}


%------------------------------------------------------------------------- 
\section{Methods}
In order to classify the sequences, we first find similarity of sequences to each other. For this purpose, we use GeCo~\cite{pratas2016efficient} to compress all sequences, considering them as references as well as targets. For measuring the similarity, normalized relative compression (NRC) is used, that can be calculated as~\cite{pratas2018comparison}
\begin{equation}
   \mathrm{NRC} (x||y) = \frac{C (x||y)}{|x|\, \log_2 |\Phi|},
\end{equation}
in which $C (x||y)$ is the information in the sequence $x$ and is obtained by compressing $x$ relatively to the sequence $y$, $|x|$ is the size of sequence $x$ and $|\Phi|$ is the cardinality of input DNA sequences, i.e. $ \mathrm{size}(\{A, C, G, T\}) = 4 $. Values of NRC falls within the range $(0, 1]$ and the more similar two sequences are, the less is this value.

GeCo works based on a mixture of finite-context models (FCMs) and extended finit-context models (XFCMs), in which mixture weights are frequently updated during the compression process, according to the performance of each probabilistic model~\cite{pratas2016efficient}. FCMs are probabilistic models that rely on Markov property and consider the $k$ most recent symbols of an information source to estimate the probability of the next symbol~\cite{hosseini2019ac}. XFCMs are probabilistic-algorithmic models that consider the occurrence probabilities stored in memory and assume that the next symbol is the one with the highest probability. Therefore, they do not consider the actual symbol in the sequence~\cite{pratas2017substitutional}.  

In the next step, we use weighted pair group method with arithmetic mean (WPGMA), which is a bottom-up hierarchical clustring method, to classify the sequences based on NRC values. The WPGMA algorithm employs a similarity matrix to construct a rooted tree (dendrogram)~\cite{sokal58a, clifford2011comparison}.


%------------------------------------------------------------------------- 
\section{Results}
The proposed method is implemented and publicly available at \url{github.com/smortezah/Clusico}, under GPLv3 license. The machine used for the tests had an 8-core 3.40 GHz Intel\textsuperscript{\scriptsize\textregistered} Core{\scriptsize\texttrademark} i7-6700 CPU with 32~GB RAM.

For the experiments, we have used 30 mitochondrial DNA (mtDNA) sequences from three groups of Actinopterygii (Ray-finned fishes), Chondrichthyes (Cartilaginous fishes) and Mammalia, that can be downloaded from \url{www.ncbi.nlm.nih.gov/nuccore}. Each groups contains 10 sequences and their sizes varies from 16,189 to 18,431 bases. These sequences are listed in Table~\ref{tab.dataset}.

\begin{table}
   \caption{Datasets used in the experiments.}
   \label{tab.dataset}
   \centering
   \vspace{3mm}
   \begin{tabular}{lll}
      \hline
      Accession & Group & Organism \\
      \hline
      NC\_005796 & Actinopterygii & \textit{Pterothrissus gissu} \\
      NC\_015337 & Actinopterygii & \textit{Canthigaster valentini} \\
      NC\_014404 & Actinopterygii & \textit{Hapalogenys nigripinnis} \\
      NC\_015823 & Actinopterygii & \textit{Diplomystes nahuelbutaensis} \\
      NC\_004449 & Actinopterygii & \textit{Gadus chalcogrammus} \\
      NC\_004701 & Actinopterygii & \textit{Eigenmannia sp. CBM-ZF-10620} \\
      NC\_016709 & Actinopterygii & \textit{Clupeoides borneensis} \\
      NC\_015544 & Actinopterygii & \textit{Horadandia atukorali} \\
      NC\_006533 & Actinopterygii & \textit{Anguilla australis schmidti} \\
      NC\_007012 & Actinopterygii & \textit{Scleropages formosus} \\
      NC\_024269 & Chondrichthyes & \textit{Lamna ditropis} \\
      NC\_021768 & Chondrichthyes & \textit{Glyphis glyphis} \\
      NC\_024862 & Chondrichthyes & \textit{Carcharhinus macloti} \\
      NC\_026696 & Chondrichthyes & \textit{Carcharhinus amboinensis} \\
      NC\_021443 & Chondrichthyes & \textit{Alopias superciliosus} \\
      NC\_022841 & Chondrichthyes & \textit{Rhinobatos hynnicephalus} \\
      NC\_022821 & Chondrichthyes & \textit{Pristis clavata} \\
      NC\_023455 & Chondrichthyes & \textit{Rhincodon typus} \\
      NC\_024110 & Chondrichthyes & \textit{Pristiophorus japonicus} \\
      NC\_027521 & Chondrichthyes & \textit{Dipturus trachyderma} \\
      NC\_027083 & Mammalia & \textit{Lynx lynx} \\
      NC\_006364 & Mammalia & \textit{Zaglossus bruijni} \\
      NC\_007629 & Mammalia & \textit{Lipotes vexillifer} \\
      NC\_025516 & Mammalia & \textit{Mustela erminea} \\
      NC\_011137 & Mammalia & \textit{Homo sapiens neanderthalensis} \\
      NC\_010299 & Mammalia & \textit{Daubentonia madagascariensis} \\
      NC\_005035 & Mammalia & \textit{Mogera wogura} \\
      NC\_008753 & Mammalia & \textit{Ursus thibetanus mupinensis} \\
      NC\_026088 & Mammalia & \textit{Megaladapis edwardsi} \\
      NC\_008417 & Mammalia & \textit{Arctocephalus pusillus} \\
      \hline
   \end{tabular}
   
\end{table}

Figure~\ref{fig.nrc}a shows similarity of different sequences (NRC values), obtained by GeCo. As is show, when a sequence is compressed relatively to itself, the NRC value will be approximately 0. These cases are shown with red squares.

Figure~\ref{fig.nrc}b demonstrates the result of classification of the sequences, which is obtained by WPGMA algorithm. The sequences in Mammalia, Chondrichthyes and Actinopterygii groups are shown with red, green and blue colors, respectively. On top of this figure, the dendrogram is plotted, which shows similarity of different sequences within each group and also, similarity of different groups. As it is show, the two groups of Chondrichthyes and Actinopterygii, that are fishes, are more similar to each other, in comparison with Mammalia.

\begin{figure*}[t!]
   \includegraphics[width=\textwidth]{fig.pdf}
   \caption{NRC.}
   \label{fig.nrc}
\end{figure*}


\bibliography{ref}
\end{document}